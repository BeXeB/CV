% ---------- Contact ----------
\newcommand{\Phone}{+36 30 555 8877}
\newcommand{\Location}{Ebes, Magyarország}

% ---------- UI text ----------
\newcommand{\TxtExperience}{Tapasztalat}
\newcommand{\TxtProjects}{Projektek}
\newcommand{\TxtSummary}{Bemutatkozás}
\newcommand{\TxtSkills}{Készségek}
\newcommand{\TxtLanguages}{Nyelvek}
\newcommand{\TxtEducation}{Tanulmányok}
\newcommand{\TxtInterests}{Érdeklődési körök}

% ---------- Sections ----------
% Experience section
% Experience 1
\newcommand{\ExpATitle}{Full Stack fejlesztő}
\newcommand{\ExpAEmployer}{Monky Shine}
\newcommand{\ExpADate}{2023. április -- jelenleg}
\newcommand{\ExpALocation}{Aalborg}
\newcommand{\ExpAContent}{
    \begin{itemize}
        \item[] \justifying
        A vállalat foglalási rendszerének fejlesztése és karbantartása Angular és .NET Core technológiák segítségével.
        Új funkciók implementálása, meglévő megoldások továbbfejlesztése, valamint a Nets fizetési rendszer
        integrálása a tranzakciók egyszerűsítése és a felhasználói élmény javítása érdekében.
        \item[] \color{black}\texttt{C\# | .NET Core | Angular | TypeScript | MSSQL | SCSS}
    \end{itemize}
}

% Experience 2
\newcommand{\ExpBTitle}{Játékfejlesztő gyakornok}
\newcommand{\ExpBEmployer}{EmptyBox}
\newcommand{\ExpBDate}{2023. január -- 2023. április}
\newcommand{\ExpBLocation}{Aalborg}
\newcommand{\ExpBContent}{
    \begin{itemize}
        \item[] \justifying
        Egy megjelenés előtt álló projekt migrálása Unity DOTS alapokra a teljesítmény és skálázhatóság javítása érdekében.
        Alaprendszerek fejlesztése, többek között egy 2D animációs rendszer Unity DOTS-hoz, valamint egy gyorsutazási rendszer.
        \item[] \color{black}\texttt{Unity | C\# | Unity DOTS}
    \end{itemize}
}

% Experience 3
\newcommand{\ExpCTitle}{Szoftverfejlesztő gyakornok}
\newcommand{\ExpCEmployer}{OleSoft}
\newcommand{\ExpCDate}{2020. augusztus -- 2020. október}
\newcommand{\ExpCLocation}{Debrecen}
\newcommand{\ExpCContent}{
    \begin{itemize}
        \item[] \justifying
        Közreműködés egy webalkalmazás fejlesztésében, amely digitalizálta a hulladéktárolók bérlésének folyamatát,
        jelentősen csökkentve az adminisztrációt és növelve a hatékonyságot.
        \item[] \color{black}\texttt{React | TypeScript | Node.js}
    \end{itemize}
}

% Projects section
% Project 1
\newcommand{\ProjAName}{Intelligens StarCraft II ügynök}
\newcommand{\ProjAContext}{Mesterképzéses szakdolgozat}
\newcommand{\ProjAContent}{
    \begin{itemize}
        \item[] \justifying
        Egy intelligens ügynök fejlesztése, amely képes StarCraft II játékot játszani.
        Az ügynök Monte-Carlo keresést használt az akciók kiválasztásához,
        valamint neurális hálókat a játékállapot elemzésére.
        \item[] \color{black}\texttt{Python | C++}
    \end{itemize}
}

% Project 2
\newcommand{\ProjBName}{Graphite – Egyedi nyelv gráfmanipulációhoz}
\newcommand{\ProjBContext}{Egyetemi projekt}
\newcommand{\ProjBContent}{
    \begin{itemize}
    \sloppy
        \item[] \justifying
        A Graphite nevű, objektumorientált programozási nyelv tervezése és implementálása,
        amely célzott műveletekkel egyszerűsíti a gráfok kezelését, valamint lehetővé teszi
        azok vizualizációját DOT integráción keresztül.
        \item[] \color{black}\texttt{C\# | DOT | Graphviz}
    \end{itemize}
}

% Project 3
\newcommand{\ProjCName}{Játékos által programozható viselkedések}
\newcommand{\ProjCContext}{Alapképzéses szakdolgozat}
\newcommand{\ProjCContent}{
    \begin{itemize}
        \item[] \justifying
        Egy olyan rendszer fejlesztése, amely lehetővé teszi a fejlesztők számára,
        hogy az objektumok kódját a játékosok számára elérhetővé tegyék,
        így azok egy egyedi, C-szerű szkriptnyelv segítségével módosíthatják a játékbeli viselkedést.
        \item[] \color{black}\texttt{C\# | Unity}
    \end{itemize}
}

% Summary section
\newcommand{\SummaryContent}{
    \begin{justify}
    \sloppy
    Frissen végzett szoftverfejlesztőként új kihívásokat és fejlődési lehetőségeket keresek.
    Szeretek új technológiákat megismerni problémamegoldáson és projektek megvalósításán keresztül,
    legyenek azok személyes vagy szakmai jellegűek.
    Szabadidőmben játékokat fejlesztek, valamint különböző eszközöket készítek hobbijaim támogatására.
    \end{justify}
}

% Skills section
\newcommand{\SkillProgramming}{Programozás}
\newcommand{\SkillAdvanced}{Haladó}
\newcommand{\SkillIntermediate}{Középhaladó}
\newcommand{\SkillBeginner}{Kezdő}
\newcommand{\SkillFrameworksTools}{Keretprogramok és eszközök}

% Languages section
\newcommand{\LangHungarian}{Magyar}
\newcommand{\LangEnglish}{Angol}

% Education section
% Education 1
\newcommand{\EduAName}{MSc számítástechnika}
\newcommand{\EduAInstitution}{Aalborgi Egyetem}
\newcommand{\EduADate}{2023. szeptember -- 2025. június}

% Education 2
\newcommand{\EduBName}{BSc szoftverfejlesztés}
\newcommand{\EduBInstitution}{University College of Northern Denmark}
\newcommand{\EduBDate}{2022. január -- 2023. június}

% Education 3
\newcommand{\EduCName}{Felsőfokú szakképzés számítástechnikából}
\newcommand{\EduCInstitution}{University College of Northern Denmark}
\newcommand{\EduCDate}{2018. szeptember -- 2020. december}

% Education 4
\newcommand{\EduDName}{Érettségi}
\newcommand{\EduDInstitution}{Mechwart András Gépipari és Informatikai Technikum}
\newcommand{\EduDDate}{2014. szeptember -- 2018. június}

% Interests section
\newcommand{\InterestsContent}{
    \begin{itemize}
    \item Játékokkal való foglalkozás és játékfejlesztés
    \item Futás és kerékpározás
    \item Olvasás
    \end{itemize}
}